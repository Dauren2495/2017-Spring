\documentclass[a4paper, 11pt]{article}

\usepackage{fullpage}
\usepackage[svgnames,x11names]{xcolor}
\usepackage{hyperref}
\hypersetup{
  breaklinks,
  colorlinks,
  linkcolor={blue!50!black},
  citecolor={blue!50!black},
  urlcolor={blue!50!black}
}
\usepackage{amsthm}
\usepackage[numbers,sort&compress]{natbib}
\usepackage{listings}
\usepackage{xspace}

\newcommand*{\Modone}{Bank\xspace}%
\newcommand*{\modone}{bank\xspace}%
\newcommand*{\Modtwo}{Shopping Mall\xspace}%
\newcommand*{\modtwo}{shopping mall\xspace}%
\newcommand*{\Modthree}{Network Operator\xspace}%
\newcommand*{\modthree}{network operator\xspace}%
\newcommand*{\Modfour}{Notary\xspace}%
\newcommand*{\modfour}{notary\xspace}%
\newcommand*{\Modfive}{Government\xspace}%
\newcommand*{\modfive}{government\xspace}%

\theoremstyle{definition}
\newtheorem{exercise}{Exercise}

\begin{document}
%%% Header starts
\noindent{\large\textbf{IS-521 Activity}\hfill
                \textbf{Team-based Software Engineering} \\
         {\phantom{} \hfill Due Date: May 24, 2017 (before midnight)} \\
%%% Header ends

\section{Activity Overview}

This is a team-based engineering activity. There are 6 teams in total
(5 teams for students and 1 team for TAs), and each team has one
public repository. Every repository will be in public, so we recommend
everyone to submit Pull Requests (PRs) to other teams.

\subsection{General Rule}

\begin{enumerate}

  \item \textbf{Email is not allowed}. Use GitHub to discuss: create
    an issue if needed, and send PRs to the repositories.

  \item Each team should set \textbf{milestones} on GitHub. Read the
    article if you are not sure how to use milestones on
    GitHub~\cite{githubissue}.

  \item Make documents everywhere in markdown, e.g., README.md.

  \item There should be a top-level makefile that builds and installs
    everything.  For example, anyone should be able to configure your
    module in a fresh new VM, provided in this course, by invoking
    only the following command:
    \begin{verbatim}git clone <your repo> && cd <repo dir> && make\end{verbatim}

  \item Every team needs to provide a set of test cases for SLA (see
    \S\ref{ss:sla}).

\end{enumerate}

\subsection{Coding Guideline}

We will give you enough freedom, but please follow the three things:
\begin{enumerate}

  \item Use maximum 80 characters per line.
  \item Do not use TAB. Use space chars instead.
  \item Extensively comment your code.

\end{enumerate}

\subsection{Issue / Pull Request (PR) Guideline}

Every student can submit PRs or issues to other teams, but the title
of your PR should follow the guideline:
%
\begin{enumerate}
%
  \item If you are creating a PR for your own team, you can use any
    title you want.
%
  \item If you are creating a PR to another team, you should indicate
    in the title that you are from another team by adding a prefix:
    ``[CROSS-TEAM]''.
%
\end{enumerate}
%
Notice that we will \textbf{\emph{NOT}} give you extra points if you do not
follow the above guideline.

\subsection{About Grading}

In this activity, every member in a team will get the same score, but
you can get individual extra points by submitting pull requests (or
issues). Not every PR will result in extra points though.  Course
staffs will carefully examine the quality of your PRs to decide
whether to give extra points.

There are two major things we expect from this activity. First, we
will see how team members make use of GitHub features such as issues,
pull requests, and milestones. Second, we will also see if each team
has good team work. For example, we may ask the following questions.
Is the work load distributed well to each team member? Do people
communicate often through GitHub?

\subsection{Test Cases for Service Level Availability (SLA)}
\label{ss:sla}

In the final CTF event of this course, we will employ an SLA checker
that periodically checks if every service is operating normally. This
means that you cannot simply shutdown your services to defend against
all possible attacks. When our SLA checker detects that one of your
service is down, we will deduct your points during the CTF event.

In this activity, you need to provide a set of test cases for each
program you developed. When we execute the service with the test cases
you provided, we should be able to cover most of the logic in your
code. In other words, we expect to achieve high \emph{code coverage}.
We recommend you to measure the code coverage with a tool such as
gcov~\cite{gcov} and coverage.py~\cite{pycov}, although this is not a
strong requirement.

Each test case in this activity is an executable program (or a script)
that can be executed by simply invoking:
\begin{verbatim}
./progname <ip> <port>
\end{verbatim}
The program takes in two command line arguments: the target IP and the
port number. When executed, the program will connect to the target
address and check if the target service is operating normally. If so,
the program terminates with an exit code of 0. Otherwise, the program
terminates with an exit code of 1. When the program cannot even
establish a connection to the target address, it returns an exit code
of 2.

Team-3 is responsible for building a library for SLA checking.
Therefore, each team should frequently check the repository of Team-3
and give feedback to them. See \S\ref{ss:modthree} to see the details
about their tasks. Teams who are building a web app should not rely on
the team-3's library: you should provide your own SLA using
\texttt{libcurl} or similar.

\subsection{Flag Updater} \label{ss:flagupdater}

Every team should develop their own flag updater program in this
activity. The main reason why we need a flag updater is to dynamically
change your flags during the CTF event. A flag updater runs as a
daemon listening on port 42. It expects to get a JSON~\cite{json}
message from the connected client.

Figure~\ref{fig:flagupdater} describes the message format. There are
three data fields: the ``signer'' field is a GitHub ID of the signer,
the ``newflag'' field is a new string to be updated for the flag, and
the ``signature'' field indicates a base64-encoded PGP signature. We
generate the signature as follows. We first concatenate the GitHub ID
of the signer with a colon `:' and the new flag. Thus, we get the
following string: \texttt{GitHubID:1234567890ABCDEF}. Next we sign the
concatenated string with the signer's private key. Finally, we
base64-encode the signature.

N.B. We will encrypt every message with the corresponding team's
public key\footnotemark[1].

Your flag updater is application-specific, meaning that you should
implement your own flag updater because the way you update your flag
depends on your application.

\begin{figure}[t]
\begin{lstlisting}[basicstyle=\ttfamily]
{
  "signer" : "TA's ID",
  "newflag": "1234567890ABCDEF",
  "signature" : "<base64 encoded string>"
}
\end{lstlisting}
\caption{An example of a flag updater message (before encryption).}
\label{fig:flagupdater}
\end{figure}

\subsection{Authentication} \label{ss:auth}

Since we have everyone's PGP public key, we will use a simple PGP-key
based authentication in this activity. First, a server asks your
github ID. When a user provides their ID, the server creates a large
random number and sends encrypts the number with the user's public
key. The server sends the encrypted data along with its public
key\footnote{
%
  You can use one of the team member's key, or you can simply create a
  new PGP key pair for the service.
%
}, and the user decrypts it with her/his private key to get the
original random number. The user now encrypts the random number with
the server's public key and send the encrypted message back to the
server. Server can verify whether this user has successfully decrypted
the message.

Throughout this document, whenever we say that you have to implement
an authentication system, it means you have to implement this
PGP-based authentication.

\section{Design}

We build a networked system in this activity. The system has 5 major
modules: (1) \modone, (2) \modtwo, (3) \modthree, (4) \modfour, and
(5) \modfive. You can assume that every module runs in a separate VM.
Each module is connected to each other within a network. Each team is
responsible for building each different module:

\begin{center}
  \begin{tabular}{ll}
    Team 1 & \Modone \\
    Team 2 & \Modtwo \\
    Team 3 & \Modthree \\
    Team 4 & \Modfour \\
    Team 5 & \Modfive \\
  \end{tabular}
\end{center}

\subsection{\Modone (Team-1)} \label{ss:modone}

Your goal here is to build a virtual banking system. The system
process runs at port 1588. When a client connect to this port, it
displays a menu that shows two options: register a new user or login.
In registration menu, the client should first authenticate him or her
with PGP key, as described in (\S\ref{ss:auth}). When the identity is
authenticated, the client can send a username and a password to create
a new user account. After the client successfully logins to the
system, it shows the following menus.

\begin{enumerate}
  \item Check balance
  \item Check transaction history
  \item Transfer
  \item Edit user info (this menu includes editing the detailed user
    information as well as removing the user account)
\end{enumerate}

You should make use of a relational database such as MySQL to store
all the necessary information for your system. We expect you to design
a schema for your database. When you design a schema, you should first
consider all possible data types that you have to handle. You should
try to be as realistic as possible. For example, you need to log every
transaction along with time stamp in order to implement the second
menu. You also need to gather information about your clients such as
email address, cellphone number, etc.
%
In case you are not familiar with database design, refer to the
following materials: \cite{dbdesign} and \cite{dbtutorial}.

We assume that every student, but not every account, initially has
1,000 won. This means, when you first create your account with your
identity, you will be automatically given 1,000 won. Later on,
however, you should not get any more money even though you create a
new account after removing an existing one.

Finally, you need to implement a flag updater (see
\S\ref{ss:flagupdater}). Your flag updater gets a JSON message,
verifies the signature of the message, and updates the email address
field of ``admin'' with the given flag string.

To summarize:
\begin{enumerate}

  \item Implement the \texttt{bank} program, which presents a
    text-based interface for banking operations.

  \item Design a realistic database schema for your system.

  \item Implement a PGP-based authentication (\S\ref{ss:auth}).

  \item Implement a flag updater.

  \item Write the document about all the menus in your banking system
    in \texttt{README.md}.

  \item Write test cases (SLA checkers) for your bank.

\end{enumerate}

\subsection{\Modtwo (Team-2)}

Your task is mainly twofold: (1) implement a web-based online shopping
mall, and (2) implement a flag updater for this module.

You have to design and implement an online shopping mall. Since this
is a web-based service, it runs on port 80 with HTTP protocol. It
should provide at least the following menus: (1) ``list'' menu shows a
list of items, (2) ``search'' menu allows a user to find an item, (3)
``shopping cart'', (4) ``view order'' menu allows a user to see the
purchased item, (5) ``message'' menu allows a user to send a message
to the admin, and (6) ``my page'' menu allows a user to see the
personal information. You have to design your own database schema for
this web application. Please refer to \S\ref{ss:modone} to see some
advice and useful links about designing your database.

In your system, each user should use the PGP-based authentication
mechanism described in \S\ref{ss:auth}.

There should be a detailed view page for each item. There should be a
button on that page to buy the item, and another button to add the
item to shopping cart. When you click the shopping cart menu, you
should be able to see the list of added items on your shopping cart.

In order to buy an item, a client should pay money via banking
transaction. In specific, first the client should enter which account
he or she will use to transfer money. Then you, as an admin, should
create a temporary account on the bank (\S\ref{ss:modone}) with your
own identity, and inform this temporary account to the user. Now you
wait for one minute, expecting money transfer from the user's bank
account previously entered.  While the system is waiting for bank
transaction, this order should appear as ``pending'' in ``view order''
menu. When you got correct amount of money transferred to your
temporary account, you should delete the account (yes you just throw
away the money for simplicity), and set the order status to be
``complete''. If the one minute timeout expires or the amount of money
transferred is not enough, you simply cancel the order and delete the
temporary account. Also, this order should not appear anymore in the
``view order'' menu.

For the ``message'' menu, the messages to admin will be stored in
your database. And you will see all of the messages in the ``my
page'' menu when you sign in.

You can put any kinds of products on your web page, but there should
be a special item called ``CTF Flag''. This item has unlimited supply:
people can buy as many as they want. The price of the item must be
1,000,000,000 won, so normal clients would not be able to buy it. If a
client purchased this item, then the client should be able to see the
content of the item from the ``view order'' menu. Of course the client
cannot see the contents while the order status is ``pending''. When
the order status becomes ``complete'', a ``view'' button should appear
right next to the ordering status. When a client clicks the ``view''
button to see the contents, the web application should read a file
located at \texttt{/var/ctf/shoppingmall.flag}, and display the flag.

Finally, you need to implement a flag updater (see
\S\ref{ss:flagupdater}). Your flag updater gets a JSON message,
verifies the signature of the message, and updates content of the file
located at \texttt{/var/ctf/shoppingmall.flag}.

Note that since the shopping mall module will be communicating with
bank system (\S\ref{ss:modone}), you should be aware of the bank
system's detailed implementation. Therefore, Team-2 should frequently
check the repository of Team-1. Or, the two teams can agree upon a
specific protocol in advance (e.g. which string will be used by the
bank system to display the menus), and follow it during the
implementation.

To summarize:
\begin{enumerate}

  \item Implement an online shopping mall web application.

  \item Design a realistic database schema for your system.

  \item Put more than 10 items to make it look nice. One of the items
    should be a ``CTF Flag'', and the price of it should be
    1,000,000,000 won. When the item is purchased, a client can see
    the content of the flag.

  \item Implement flag updater.

  \item Write the document about all the menus in your shopping mall
    system in \texttt{README.md}.

  \item Write test cases (SLA checkers) for your web app.
\end{enumerate}

\subsection{\Modthree (Team-3)} \label{ss:modthree}

There are three main tasks here: (1) building a library for SLA, (2)
building a SLA logger, and (3) implement a web app that visualizes an
SLA log.

First, you need to build an SLA checker library in C. Recall from
\S\ref{ss:sla}, SLA checker is a program that takes as input a target
address and a port number, and outputs an exit code of 0, 1, or 2
depending on the service availability. The library needs to provide at
least the following functionalities.

\begin{enumerate}
%
  \item \texttt{int openTCPSock(char *IP, unsigned short port)}\\
    This function returns a TCP socket handle that can be used to
    communicate with the service running in the provided IP and port
    number. The function should return a negative value to indicate an
    error. It would be desirable to return different negative value
    for different error, to provide useful information to the library
    user. Describe the specification of this return value in your
    \texttt{README.md} file.
%
  \item \texttt{int openUDPSock(char *IP, unsigned short port)}\\
    This function is the same as \texttt{openTCPSock} except that it
    opens a UDP socket.
%
  \item \texttt{void closeSock(int sock)}\\
    This function closes the given socket handle.
%
  \item \texttt{ssize\_t recvMsgUntil(int sock, const char* regex,
    void* buf, size\_t n)}\\
    This function reads in maximum \texttt{n} bytes from the given
    socket. The received data is stored into a buffer pointed by
    \texttt{buf}. The function must return as soon as the data
    received so far matches the provided regular expression
    \texttt{regex}. The return value is the length of data received at
    the point of return. If an error occurred while receiving the
    data, return a negative integer value. Likewise, describe the
    return value's specification in \texttt{README.md} file.
%
  \item \texttt{int sendMsg(int sock, const char* buf, size\_t n)}\\
    This function sends a string to the connected socket. It returns a
    negative value when there is an error. You should document its
    specification in \texttt{README.md} file.
%
  \item \texttt{int handshake(int sock, const char* ID, const char*
    privKeyPath, const char* passPath, const char* successMsg)}\\
    This function performs PGP-based authentication described in
    \S\ref{ss:auth}. \texttt{ID} is the Github ID of the client.
    \texttt{privKeyPath} is a path to PGP private key file, and
    \texttt{passPath} is a path of the passphrase file for the private
    key. \texttt{successMsg} is the expected string from server if the
    authentication succeeded. Decide the return value for each
    possible situation, and document it in \texttt{README.md} file.
%
\end{enumerate}

You also need to write a test case (a program) for testing a DNS
server using your library. Imagine there is a DNS server in your own
private network. Your goal is to check whether the DNS server is
working or not. Your program should read in "./expect.csv" file which
consists of two columns: the first column is a domain name to query
(e.g. "bank.com"), and the second column is the IP expected as
response (e.g. "10.0.0.1"). You can send a raw DNS packet to the
server and parse the returned message. Write your test case and run it
with your SLA checker to see if it works.

Another thing to implement is a logger that periodically spawns a
series of test cases provided by the user, and record the availability
of a set of target services. The log is in a CSV
format~\cite{csvwiki}. There are four columns in total in the log: (1)
UNIX timestamp, (2) target IP address, (3) target port address, (4)
status. The status field is represented with either ``up'' or
``down''. The usage of the logger should be:
%
\begin{verbatim}
./logger [ip addr] [port] [dir containing test cases] [logfile]
\end{verbatim}
%
The third argument of the logger is a path to a directory that
contains a list of test cases. The fourth argument is a path to the
output log file. The logger uses a \texttt{seccomp} filter to sandbox
the execution environment. You have to set your criteria, and make a
document in your \texttt{README.md} file. Type \texttt{man seccomp}
for more information.

You also need to provide a web interface that shows a line graph where
X-axis is time, and Y-axis is the availability of a service. The web
application takes in a log file as input from a user, and shows the
corresponding line graph. Each service corresponds to a line in the
graph. This page should be only usable to its team members. To
authenticate the identity of the team members, implement a PGP-based
authentication described in \S\ref{ss:auth}.

Finally, you need to implement a flag updater (see
\S\ref{ss:flagupdater}). Your flag updater gets a JSON message,
verifies the signature of the message, and updates content of the file
located at \texttt{/var/ctf/sla.flag}.

To summarize:
\begin{enumerate}

  \item Implement SLA checker library.

  \item Implement SLA logger (with seccomp).

  \item Implement a web-based log visualizer.

  \item Implement flag updater.

  \item Write test cases (SLA checkers) for a DNS server and your log
    visualizer service.

  \item Write the document about your design and implementation in
    \texttt{README.md}.

\end{enumerate}

\subsection{\Modfour (Team-4)} \label{ss:notary}

Your goal is twofold here: (1) build a notary program, and (2) build a
launcher program that only executes a signed program by the notary.

Notary runs at port 8000. It simply takes in a file as input from a
client, and returns the file along with a signature of the file. Of
course, only an authenticated user should be able to get a signature
from the notary. To check the authenticity, it uses PGP keys that we
used in the class. In particular, the notary program takes in one
command line argument: a path to a directory that contains PGP public
keys. It only accepts files that are signed by an owner of the public
keys.
%
The notary program should log all the events to one or more files in
the directory: \texttt{/var/log/notary/}.

We do not specify the output format in detail, but it should be in
JSON, and you should detail the format in your document (README.md).

The second program to write is a launcher, which takes in a certified
program from the notary, and executes it only when the program is
certified. Since this can be potentially vulnerable, you want to
create a chrooted environment for executing the program. The launcher
takes in a JSON-based request at port 8001, but it only accepts
packets coming from a local network. The JSON has the following form:

\begin{lstlisting}
{
  "name": "name of the program",
  "body": "base64 encoded program executable"
}
\end{lstlisting}

To only accept packets from the local network, the launcher takes in
two command line arguments that specifies the range of allowed IPv4
addresses. For example, if 10.0.0.1 and 10.0.0.10 are given as command
line arguments, then it means we can only accept network packets from
the IP addresses from 10.0.0.1 to 10.0.0.10. Your launcher should
analyze the IP layer packets to check where the messages are coming
from.

Additionally, your launcher should perform a system call tracing using
the ptrace system call. In particular, you want to prevent a program
to spawn an \texttt{execve} system call, because it may spawn an
unwanted shell to an attacker.

Finally, you need to implement a flag updater (see
\S\ref{ss:flagupdater}). Your flag updater gets a JSON message,
verifies the signature of the message, and updates content of the file
located at \texttt{/var/ctf/notary.flag}.

To summarize:
\begin{enumerate}

  \item Implement a notary program.

  \item Write your format specification for returning messages from
    the notary program.

  \item Implement a launcher program.

  \item Implement flag updater.

  \item Write test cases (SLA checkers) for your notary and launcher.

  \item Write the document about your design and implementation in
    \texttt{README.md}.

\end{enumerate}

\subsection{\Modfive (Team-5)}

You need to design two things here: a web service, and a bot that
automatically browses the web.

Government is a virtual object. Government runs its own homepage at
port 80. The homepage has several static web pages as well as dynamic
web pages. Static web pages include some information about the
government. Dynamic web pages are basically to host a web-based
bulletin board. Anyone should be able to create a user account and
write a message to the board. All the information including the user
account information and messages should be written in a database. Do
\emph{not} use existing solutions such as zeroboard for this project.
In the bulletin board, we use the PGP-based login mechanism described
in \S\ref{ss:auth}.

A logged in user should be able to view a bulletin board, and there
should be at least six operations in the bulletin board: (1) ``list''
operation shows a list of postings, (2) ``write'' operation writes a
new posting to the board, (3) ``read'' operation reads the contents of
a specific posting, (4) ``delete'' operation removes a specific
posting from the board, (5) ``edit'' operation edits the contents of a
specific posting from the board, and (6) ``authenticate'' operation
enables a user to notarize with the notary service
(\S\ref{ss:notary}).

Notice that there are two types of users for the bulletin board: (1)
normal user, and (2) notarized user. Normal users cannot write a post
that contains Javascript, whereas notarized users can. To get
notarized by the notary service, a user needs to supply a signed JSON
file from the notary. The JSON file should contain the user's ID and
the user's public key. You have to design a web interface that a valid
user can get notarized after login.

You also need to design and implement a browsing bot. This bot
launches a real web browser such as firefox and chrome. It parses the
currently loaded web page, and clicks links of your interest. We
expect your bot to read \emph{every} posting from the bulletin board.
You can write such an application using WebDriver~\cite{webdriver}.
The web browser used by the bot should have a cookie originated from a
virtual web site called \texttt{bank.com}. The cookie contains a
record, which has the name ``flag'', and the value of the flag is a
32-byte random ASCII string.

Finally, you need to implement a flag updater (see
\S\ref{ss:flagupdater}). Your flag updater gets a JSON message,
verifies the signature of the message, and updates content of the
cookie for the domain \texttt{bank.com}. Specifically, it modify the
``flag'' value of the cookie.

To summarize:
\begin{enumerate}

  \item Implement a bulletin board web application.

  \item Implement a government homepage that has a bulletin board.

  \item Implement a bot with WebDriver.

  \item Implement flag updater.

  \item Write test cases (SLA checkers) for your web app.

  \item Write the document about your design and implementation in
    \texttt{README.md}.

\end{enumerate}

\bibliography{references}
\bibliographystyle{plainnat}

\end{document}
