\documentclass[a4paper, 11pt]{article}

\usepackage{fullpage}
\usepackage{hyperref}
\usepackage{amsthm}
\usepackage[numbers,sort&compress]{natbib}

\theoremstyle{definition}
\newtheorem{exercise}{Exercise}

\begin{document}
%%% Header starts
\noindent{\large\textbf{IS-521 Activity}\hfill
                \textbf{Backdoor} \\
         {\phantom{} \hfill Due Date: March 20, 2017 (before class)} \\
%%% Header ends

\section{Activity Overview}

The primary goal of this activity is to learn how backdoors work by
creating your own. You will build an interpreter, and programs based
upon a simple language called ``Mini Language''.

\section{Mini Language}

We first introduce Mini Language, a simple register-based language.
This language consists of 12 instructions including \texttt{halt},
\texttt{load}, \texttt{store}, etc.

% Each instruction consumes 4 bytes: 1 byte for opcode, and 3 bytes for
% operands. If an opcode such as \texttt{halt} does not require any
% operand,

\bibliography{references}
\bibliographystyle{plainnat}

\end{document}
