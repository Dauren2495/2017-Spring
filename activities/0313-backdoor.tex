\documentclass[a4paper, 11pt]{article}

\usepackage{fullpage}
\usepackage{hyperref}
\usepackage{amsthm}
\usepackage[numbers,sort&compress]{natbib}
\usepackage{listings}

\theoremstyle{definition}
\newtheorem{exercise}{Exercise}

\begin{document}
%%% Header starts
\noindent{\large\textbf{IS-521 Activity}\hfill
                \textbf{Backdoor} \\
         {\phantom{} \hfill Due Date: March 20, 2017 (before class)} \\
%%% Header ends

\section{Activity Overview}

The primary goal of this activity is to learn how backdoors work by
creating your own. You will have to build an interpreter and programs
based upon a simple language called ``Mini Language''. You will also
introduce a backdoor to your own interpreter.

\section{Mini Language}

We first introduce Mini Language, a simple language that runs on a
register-based virtual machine. This language consists of 14
instructions including \texttt{halt}, \texttt{load}, \texttt{store},
etc. Each instruction starts with an opcode followed by at most 3
operands. There is at least one space character between an opcode and
the first operand. Each operand is separated by a comma. An operand
can be either a register or an immediate. An immediate can represent a
number or an address. Finally, we assume that each process is given a
8k heap memory starting at the address zero to 8192. This address
space does not overlap with code, which is also started from address
zero.

\subsection{Operands}

\paragraph{Register (reg).} There are 256 available registers in our
language. We denote a register by `r' followed by a register ID:
\texttt{r0, r1, \ldots, r255}. Each register is 32-bit (4-byte) long.

\paragraph{Immediate (imm).} An immediate operand indicates either an
integer number or an address depending on the opcode. We use a simple
number representation to denote an immediate.

\subsection{Opcode}

There are 14 opcodes in total. Your VM should terminate immediately if
it encounters an unknown opcode, i.e., opcodes other than the 14
opcodes defined, with an error message.

\begin{description}

  \item [\texttt{halt}]~\\
    %
    This instruction does not require any operand. This instruction
    simply terminates the current process.

  \item [\texttt{load (reg), (reg)}]~\\
    %
    This instruction loads a 1-byte value from memory located at
    address (the second register) into the first register. The high 24
    bits are zeroed out. For example, \texttt{load r0, r1} means: load
    a 8-bit value from the address at \texttt{r1} into \texttt{r0}.

  \item [\texttt{store (reg), (reg)}]~\\
    %
    This instruction stores a 1-byte value of the second register to
    memory located at address (the first register). For example,
    \texttt{store r1, r2} means: store a byte value (the lower 8-bit
    of \texttt{r2}) to memory address at \texttt{r1}.

  \item [\texttt{move (reg), (reg)}]~\\
    %
    This instruction moves a 32-bit value from a register to another.
    Both operands should be a register, but not an immediate. For
    example, \texttt{move r1, r0} means: move a 32-bit value from
    \texttt{r0} to \texttt{r1}.

  \item [\texttt{puti (reg), (imm)}]~\\
    %
    This instruction moves an 8-bit immediate value to a register. The
    upper 24-bit of the destination register is zeroed out.

  \item [\texttt{add (reg), (reg), (reg)}]~\\
    %
    This instruction adds two values from the second and the third
    register, and store the result to the first register. For example,
    \texttt{add r0, r1, r2} means: add the value of \texttt{r1} and
    \texttt{r2} and store the result to \texttt{r0}. When an overflow
    happens the result will be wrapped around the max integer
    ($2^{32}-1$). In other words, \texttt{0xffffffff + 0x1 = 0x0}.

  \item [\texttt{sub (reg), (reg), (reg)}]~\\
    %
    This instruction subtracts two values from the second and the
    third register, and store the result to the first register. For
    example, \texttt{sub r0, r1, r2} means: subtract the value of
    \texttt{r2} from \texttt{r1} and store the result to \texttt{r0}.
    Overflow conditions are handled the same as in addition
    (\texttt{add}).

  \item [\texttt{gt (reg), (reg), (reg)}]~\\
    %
    This instruction compares two values of the second and
    the third register, and store the comparison result to the first
    register: 1 if the second register is greater than the third
    register; 0 if otherwise.

  \item [\texttt{ge (reg), (reg), (reg)}]~\\
    %
    This instruction compares two values of the second and
    the third register, and store the comparison result to the first
    register: 1 if the second register is greater than or equal to the
    third register; 0 if otherwise.

  \item [\texttt{eq (reg), (reg), (reg)}]~\\
    %
    This instruction compares two values of the second and
    the third register, and store the comparison result to the first
    register: 1 if the second and the third register have the same
    value; 0 if otherwise.

  \item [\texttt{ite (reg), (imm), (imm)}]~\\
    %
    This instruction (if-then-else) checks the value of the first
    register, and changes the instruction pointer to have a value
    (either of the second and the third operand). When the register
    value is greater than zero, then the instruction pointer will have
    the value of the second operand in the next execution. If the
    register value is equal to zero, then the instruction pointer will
    have the value of the third operand. The instruction pointer can
    have an address ranging from 0 (the first instruction) to $N$,
    where $N$ is the number of instructions in the code. See
    \texttt{jump} instruction for more details.

  \item [\texttt{jump (imm)}]~\\
    %
    This is a jump instruction that changes the instruction pointer.
    The immediate value indicates an address to be executed in the
    next instruction. The target address is an absolute index of an
    instruction in the code. For example, \texttt{jump 2} means the
    next instruction pointer should point to the third instruction of
    the code. If the target instruction does not exist, the VM should
    produce an error message and terminate immediately.

  \item [\texttt{puts (reg)}]~\\
    %
    This instruction outputs an ASCII string (terminated by a NULL
    character) located at memory address (reg) to the screen. The
    semantics of puts instruction is equivalent to the puts function
    in LIBC except that it does not append a newline character at the
    end.

  \item [\texttt{gets (reg)}]~\\
    %
    This instruction takes an ASCII string (terminated by a NULL
    character) as input from STDIN and store the string to memory
    address (reg). The semantics of gets instruction is equivalent to
    the gets function in LIBC. Specifically, reading stops when a
    newline character is found, and a NULL character is appended to
    the end of the string.

\end{description}

\subsection{Bytecode}

Each instruction consumes 4 bytes when it is compiled: the first byte
indicates the opcode, and the rest three bytes represent the operands.
We always use 4 bytes for an instruction even though it does not
require three operands. Unused operands can have any value in the
binary form. For example, both \texttt{0x00000000} and
\texttt{0x00111111} are considered as a \texttt{halt} instruction.
Notice the first byte (the operand) should be zero, meaning a halt
opcode, in both cases.

\section{Compiler}

We provide a compiler written in OCaml. This compiler takes in a
program written in Mini Language, and outputs a bytecode. The compiler
is deliberately simple, e.g., it does not even type check. However, it
should be enough to be used in this activity. If you can improve this
compiler, and create a pull request, we will give you up to +10 extra
credit.

\section{Interpreter (VM)}

We provide starter code for creating an interpreter for Mini Language.
Your interpreter takes in a program (in byte code) and executes the
program starting from the very first instruction of the program. When
a program is loaded, your interpreter allocates a heap memory range
from 0 to 8192. A program written in Mini Language can use hard-coded
addresses to access the heap memory. When accessing a memory region
outside of the predefined area, your VM should terminate immediately
with an error message.

For your information, the current starter code is derived from a
project called Mini-VM~\cite{minivm}.

\section{Login Program}

You need to create a simple program written in Mini Language. The
pseudocode of this program is given in Listing~\ref{lst:login}.

\begin{lstlisting}[language=C,
                   caption={Pseudocode for Login Program.},
                   label=lst:login,
                   basicstyle=\footnotesize\ttfamily]
void main(void)
{
  char* buf = malloc( /* some amount */ );
  printf( "User: " );
  gets( buf );
  if ( strcmp( buf, "user" ) == 0 ) {
    printf( "Password: " );
    gets( buf );
    if ( strcmp( buf, "password" ) == 0 ) {
      puts( "Success" );
    } else {
      puts( "Failure" );
    }
  } else {
    puts( "Failure" );
  }
  exit( 0 );
}
\end{lstlisting}

\section{Test Program}

You need to write a simple test program in Mini Language. This is to
demonstrate that your VM is running correctly. We ask you to write a
markdown description for your test program that includes a pseudocode
(in C). The markdown description should clearly show what is the
expected input/output of the program.

\section{Installing Backdoor}

Install a backdoor to the interpreter you created. This backdoor
allows you to bypass the authentication process of the login program.
Specifically, when you use a user ID ``superuser'', then this backdoor
allows you to bypass the check, and the program will immediately
return the ``Success'' message without providing any password. Your
goal is to create a backdoor in the interpreter, which can detect your
login program, and surreptitiously change the control flow of the
program when a user ID ``superuser'' is given.

\section{Final Deliverables}

We expect you to push your source code to your own public repository
(backdoor-your\_id). You \emph{must} follow the file naming convention
described below. \textbf{If otherwise, course staffs will immediately
deduct half of the available score}.

\begin{enumerate}

  \item \textbf{./interpreter/} directory contains source code for
    your Mini Language Interpreter (VM).

  \item \textbf{./login/login.mini} is a login program written in Mini
    Language. The program should be semantically equivalent to the
    pseudocode given in Listing~\ref{lst:login}.

  \item \textbf{./test/test.mini} file is a test program written in
    Mini Language. We expect to see a unique test program for each
    student.

  \item \textbf{./test/test.md} file is a description of the test
    program. This write-up includes a pseudocode of your test program
    as well as a brief description of your program. In the
    description, you need to clearly mention the expected input/output
    of the program.

  \item \textbf{./backdoor/} directory contains source code for your
    Mini Language Interpreter (VM) with your own backdoor. This
    directory should have the \emph{same} list of files as in the
    ./interpreter/ directory.

  \item \textbf{./compiler/compiler.ml} file is a simplistic version
    of compiler for Mini Language. You can \emph{optionally} improve
    this compiler to get extra points (up to +10). If you made a
    modification on this file, please clearly indicate in the
    \texttt{./README.md} file that you modified the compiler.

  \item \textbf{./README.md} file contains a write-up. This write-up
    simply lists what you did for each directory and file, and what
    you learned from this activity.

\end{enumerate}

\bibliography{references}
\bibliographystyle{plainnat}

\end{document}
