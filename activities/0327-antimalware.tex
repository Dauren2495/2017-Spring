\documentclass[a4paper, 11pt]{article}

\usepackage{fullpage}
\usepackage{hyperref}
\usepackage{amsthm}
\usepackage[numbers,sort&compress]{natbib}
\usepackage{listings}

\theoremstyle{definition}
\newtheorem{exercise}{Exercise}

\begin{document}
%%% Header starts
\noindent{\large\textbf{IS-521 Activity}\hfill
                \textbf{Anti-malware} \\
         {\phantom{} \hfill Due Date: April 5, 2017 (before class)} \\
%%% Header ends

\section{Activity Overview}

In this activity, you will build your own malware scanner. We provide
malware samples from the previous activities. Since the malware
samples (although it is written by different authors) have the same
semantics, and we can call them as the ``myworm'' malware family.
Notice, each malware binary should be syntactically different from
each other because they are written individually by each student.

Your goal in this activity is threefold: (1) design a scanner that
detects the ``myworm`` malware family using a library called
YARA~\cite{yara}; (2) create a variant of your previous malware, and
try to bypass other students' signatures; and (3) create a benign
sample that does semantically similar activity as myworm in order to
confuse other students' scanners.

You want to make your signature from (1) to be as strong as possible,
so that it can detect other's malware variants in (2) as well as
yours. We will run your scanner against 60 malware samples (30 from
the previous activity, and 30 from this activity), and 30 benign
samples (from this activity). We will give extra points (up to +10) to
a student who designed the most powerful scanner that detects the most
of the malware samples with the least false alarms.

\section{Malware Scanner Design} \label{s:scanner}

We provide starter code for your malware scanner~\cite{scanner}. Your
scanner will run in two modes. First, when a user does not specify any
directory to scan, it will simply check all the currently running
processes, and obtain the absolute path of each of the running
processes, and scan the corresponding files. Second, when a user
specifies at least one directory to scan, then you simply recursively
enumerate and scan all the files in the directory. You should use
YARA's APIs to scan files.  Please refer to the YARA manual
online~\cite{yaramanual} for more information. Writing a rule (file)
for YARA is also straightforward.  Please search google for ``yara
rule'' to see related documents.

We detail the specification of the scanner as follows.
%
\begin{enumerate}

  \item Your scanner takes in a command-line argument \texttt{-r} to
    take in a rule file. If a rule file is not given, your scanner
    should emit an error message, and terminate the process.

  \item Your scanner takes in a list of directory names. If the list
    is empty, then the scanner simply checks all the binary files that
    are currently running. To see how to check the currently running
    processes in C, you may want to refer to the source code of ``ps''
    command:
    \url{https://github.com/mmalecki/procps/blob/master/ps/display.c}.

  \item When you found a match, your scanner should print out the
    following string including the file name:
    \begin{verbatim}File Name: FOUND!\end{verbatim}

  \item When a file does not match with any of your rules, your
    scanner should print out the following string:
    \begin{verbatim}File Name: Not matched.\end{verbatim}

\end{enumerate}

\section{Writing a Variant}

Your goal is to write a malware variant that can bypass
signature-based matching. There are several techniques that you can
use including function reordering, register reordering, dummy code
insertion, etc. You can use any creative techniques in your mind to
make a strong malware variant. Your first goal should be to design a
malware variant that can bypass your original signature wrote in
\S\ref{s:scanner}. It is allowed to use malware code written by other
peers if you could not finish the previous activity.

You can also directly modify the binary code itself using a hex editor
or similar. But you should not use existing packers such as UPX to
pack your binary. You should write your own variant, and you need to
clearly explain what you did in \texttt{Variant.md}. Course staffs can
deduct points if your explanation is not sufficient.

\section{Writing a Strong Signature}

Now that you have created a malware variant, your final goal is to
write a strong signature that can detect variants of ``myworm''
family. We say a program is a member of ``myworm'' family, if the
program scans for vulnerable network ports, and exploits the vulnet
vulnerability to execute an arbitrary logic.

In the Git repository, you should only include the final signature
that you created in this step. Notice your scanner with your final
signature will be used against all the malware variants written by
your peers. Try to guess what people would do to make variants, and
make your signature as strong as possible.

Of course, you should think both false positives and negatives to
design your rules. For example, you can simply write a rule that
always raise an alarm for any given file. This rule can detect all the
malware samples, but any benign sample will raise a false alarm. We
will open all 30 malware samples on the next (Wed.) Lab session.

\section{Writing a Benign Sample}

We also ask you to write a file that can potentially be detected as
``myworm'' family, even though it does not do any harm. The idea is to
create a malicious-looking but benign samples to deceive the scanners
written by other peers. By benign, we mean that your program should
not leak any information from the vulnerable server other than the
fact that the server is running a vulnerable vulnet server. Imagine a
vulnerability scanner, which is benign, but may look similar to
myworm.

We will run your scanner against the benign samples from other
students. We will give extra points to students who cheated the most
number of scanners.

\section{Deliverables}

We expect you to push your source code to your own private repository
(av-your\_id). You \emph{must} follow the file naming convention
described below. You are allowed to add extra files that are not
specifically mentioned in the list below such as 3rd-party libraries
or configuration files, but you should always include the followings.
\textbf{If otherwise, course staffs will immediately deduct half of
the available score}.

\begin{enumerate}

  \item \textbf{./Makefile} file. We should be able to build your
    scanner simply by typing ``make''. The produced scanner binary
    should be placed in \textbf{./build/scanner}.

  \item \textbf{./scanner.c} file contains the main scanner source.
    You can add additional files in the same directory, but you should
    mention them in the \texttt{README.md}.

  \item \textbf{./myworm.yar} file contains your final rule(s) to
    detect ``myworm'' malware family.

  \item \textbf{./benign/} directory contains source code for your
    benign sample.

  \item \textbf{./benign/Makefile} file should produce your benign
    sample in \texttt{./build/benign.bin}. This file should be invoked
    in the top-level Makefile.

  \item \textbf{./benign/Benign.md} file details how you created a benign
    sample.

  \item \textbf{./variant/variant.bin} file is a variant of myworm
    (binary) you generated.

  \item \textbf{./variant/Variant.md} file details how you manipulated
    your malware to bypass peers' signatures.

  \item \textbf{./README.md} file contains a write-up. This write-up
    simply lists what you did for each directory and file, and what
    you learned from this activity.

\end{enumerate}

\bibliography{references}
\bibliographystyle{plainnat}

\end{document}
