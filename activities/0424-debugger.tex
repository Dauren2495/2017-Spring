\documentclass[a4paper, 11pt]{article}

\usepackage{fullpage}
\usepackage{hyperref}
\usepackage{amsthm}
\usepackage[numbers,sort&compress]{natbib}
\usepackage{listings}

\theoremstyle{definition}
\newtheorem{exercise}{Exercise}

\begin{document}
%%% Header starts
\noindent{\large\textbf{IS-521 Activity}\hfill
                \textbf{Debugger} \\
         {\phantom{} \hfill Due Date: May 3, 2017 (before class)} \\
%%% Header ends

\section{Activity Overview}

In this activity, you build your own custom value tracer with the
ptrace debugging API~\cite{ptrace}. Before anything else, remember to
read the manual: \texttt{man ptrace}. Your value tracer takes in an
executable and a CSV file as input, and prints out a sequence of
register or memory values specified in the CSV file by running the
executable.

Your value tracer takes in an executable file as the first argument,
and a Comma-Separated Values (CSV)~\cite{csvwiki} file as the second
argument. For example, the following command loads \texttt{/bin/ls}
with a CSV file \texttt{a.csv}:
%
\begin{verbatim}./tracer /bin/ls ./a.csv\end{verbatim}

\section{CSV File}

The CSV file consists of two columns: the first column indicates
breakpoint addresses in hexadecimal, and the second column represents
tracing targets. A breakpoint address always starts with the prefix
`0x'.  Tracing targets are either a register or a memory address. Any
tracing target starts with `0x' is considered to be a memory address.
If tracing target is an integer between 0 and 8, it means a register
ID specified in Table~\ref{tab:regid}. If otherwise, it is considered
to be an error: your tracer should terminate with an error message.

Consider the following CSV line: \texttt{0x1000,0x4000}.
%
This line means that your value tracer needs to print out the 4-byte
value at 0x4000 when the program counter of the program under test is
at 0x1000. Notice, we assume that your value tracer always prints out
4-byte values for simplicity.

You can consider each line of the CSV file as a breakpoint. You cannot
have duplicate breakpoint addresses in the CSV file, but you do not
need to have an ordered addresses in the CSV. For example, suppose we
have the following CSV file with 4 lines.
%
\begin{verbatim}
0x1000,0
0x1009,2
0x1003,1
0x1002,0
\end{verbatim}
%
In the above case, your value tracer should insert four distinct
breakpoints at address 0x1000, 0x1002, 0x1003, and 0x1009. Notice the
program under test can execute the same address more than once, e.g.,
when it has a loop. Suppose your program executes an instruction at
address 0x1000 for five times. In this case, your value tracer should
print out the value of eax exactly 5 times.

When the given CSV has duplicate breakpoint addresses, then your value
tracer should emit an error message, and immediately exit. For
example, given the following CSV:
\begin{verbatim}
0x1000,0
0x2000,5
0x1000,0x2000
\end{verbatim}
%
Your value tracer should terminate, because the address 0x1000 appears
twice in the CSV. Similarly, when there is an invalid address, your
tracer should exit with an error message.

\begin{table}[h]
  \centering
  \begin{tabular}{lc}
    ID & Register Name \\\hline
    0 & eax \\
    1 & ebx \\
    2 & ecx \\
    3 & edx \\
    4 & esi \\
    5 & edi \\
    6 & ebp \\
    7 & esp \\
    8 & eip \\\hline
  \end{tabular}
  \caption{Register ID}
  \label{tab:regid}
\end{table}

\section{Value Tracer}

Your value tracer binary should have the name \texttt{tracer} when it
is compiled. Your value tracer inserts software breakpoints (0xcc) at
the addresses specified in the CSV file, and run the program under
ptrace. Every time a breakpoint hits, your value tracer restores the
original instruction and executes it in order to preserve the
semantics of the original binary.

\paragraph{Output.} The output is a CSV of two columns: (1) the
current address (\texttt{EIP}), and (2) the traced value (4-byte
register or memory value). Both values should be prefixed with `0x'.
We will deduct the half of available points if you do not follow the
convention.

\paragraph{Evaluation.} To evaluate your value tracer, we will prepare
for a list of binary and CSV pairs, and run your value tracer on them
to see if it can correctly print out the values.

\section{Deliverables}

We expect you to push your source code to your own private repository
(tracer-your\_id). You \emph{must} follow the file naming convention
described below. You are allowed to add extra files that are not
specifically mentioned in the list below such as 3rd-party libraries
or configuration files, but you should always include the followings.
\textbf{If otherwise, course staffs will immediately deduct half of
the available score}.

\begin{enumerate}

  \item \textbf{./Makefile} file. We should be able to build your
    value tracer simply by typing ``make''. We will use the VM that we
    provided in the course to compile your value tracer. If there is
    any necessary dependencies to build your value tracer, your
    Makefile \emph{must} automatically install them. The produced
    binary should be placed in \textbf{./build/tracer}.

  \item \textbf{./src/} directory contains the source of your value
    tracer. You should use C for this activity. N.B. Points will be
    deducted for the lack of necessary comments. Please make your code
    readable.

  \item \textbf{./README.md} file contains a write-up. This write-up
    simply lists what you did for each directory and file, and what
    you learned from this activity.

\end{enumerate}

\bibliography{references}
\bibliographystyle{plainnat}

\end{document}
