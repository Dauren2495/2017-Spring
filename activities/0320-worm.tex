\documentclass[a4paper, 11pt]{article}

\usepackage{fullpage}
\usepackage{hyperref}
\usepackage{amsthm}
\usepackage[numbers,sort&compress]{natbib}
\usepackage{listings}

\theoremstyle{definition}
\newtheorem{exercise}{Exercise}

\begin{document}
%%% Header starts
\noindent{\large\textbf{IS-521 Activity}\hfill
                \textbf{Worm} \\
         {\phantom{} \hfill Due Date: March 27, 2017 (before class)} \\
%%% Header ends

\section{Activity Overview}

In this activity, you need to build your own worm. The worm exploits a
vulnerability introduced by a backdoor installed in a \emph{vulnet},
which is a simple telnet-like service. Your goal is to design a worm
that can automatically scan a specified network range, and can
propagate through the network by exploiting the vulnerability. You
will be working on a \emph{private repository} for this activity,
since the outcome of this activity can be security-sensitive. You can
use any language to write your worm, although we strongly recommend
you to use C. We do not provide any starter code for this activity.
Please frequently check the course git repository to get updates on
the activity. We also recommend you to ``watch'' the original repo.

\section{The vulnet} \label{s:vulnet}

The vulnet service provides a virtual terminal connection to users. It
uses a password-based authentication, and spawns a remote
\texttt{bash} shell to the authenticated users. We provide source code
for vulnet~\cite{vulnet}. You should be able to spot an obvious
backdoor routine from the source code. Specifically, when a user types
in ``superuser'' as a user name, the program immediately logs the user
in without taking a password. The vulnet uses TCP port number 2323 by
default.

\section{The Worm}

The primary goal of this activity is to design your own worm. The worm
should exploit the vulnerability of vulnet introduced by the backdoor
described in the previous section.

\subsection{The Operation}

The worm runs in the following manner:

\begin{enumerate}

  \item It scans a user-specified network range. For every IP in the
    range, it checks if TCP port number 2323 can be connected. Recall
    the vulnerability described in \S\ref{s:vulnet}.

  \item Once connected, it checks if this port is used as vulnet, and
    tries to log in to the server by exploiting the vulnerability. In
    case the vulnet server does not have a backdoor, it simply goes to
    the next target.

  \item If it can be sure that the server has the backdoor, it then
    tries to obtain the valid user name and the password. Hint:
    \texttt{/proc/} filesystem (procfs) can be useful.
    \label{step:infer}

  \item If there exists a file in \texttt{/tmp/myworm}, the worm
    assumes that the machine has already got infected, and goes to the
    next step. Otherwise, it stores the binary of itself (the worm
    binary) to \texttt{/tmp/myworm}. For the purpose of this exercise,
    we just close the connection at this point without running the
    worm binary, but in reality, you would run the binary. N.B., we
    will deduct points if you execute the worm after propagation.

  \item It re-connects to the server with the obtained user name and
    the password from Step~\ref{step:infer}. If the login works, it
    records the IP address of the vulnerable host along with a user
    name and a password. Otherwise, it simply records the IP address
    with a user name ``superuser''.

  \item This process iterates until the scanning process ends. The
    final output is a CSV (Comma-Separated Values) file that has three
    tuples per each line: IP, user name, password. An example would
    be:
    \begin{verbatim}
    123.0.0.1,admin,secret
    42.42.42.42,superuser,
    ...
    \end{verbatim}

\end{enumerate}

\subsection{Specification}

The worm has the following specification.

\begin{enumerate}

  \item The worm binary has the name \texttt{myworm} (all lowercase).

  \item The worm binary takes in three command-line arguments: the
    first and the second argument specify the lower and the upper ip
    address (IPv4) respectively. The third argument specifies the name
    of an output file (the default should be \texttt{./slaves.csv}).
    For example, if the first and the second argument are
    $192.168.0.0$ and $192.168.255.255$, then the worm will scan
    65,536 hosts from the given range.
    %
    The third argument is optional, and if it is not given, the
    default file name is used. For example, the following command will
    scan 256 hosts, and store the list of vulnerable hosts in the
    \texttt{./slaves.csv} file.
    \begin{verbatim}
    ./myworm 192.168.0.0 192.168.0.255
    \end{verbatim}

  \item When an incorrect range is given as input, the program should
    return an error message, and terminates the process. For example,
    the following is a valid command.
    \begin{verbatim}
    ./myworm 192.168.0.0 192.168.0.0
    \end{verbatim} But, the following is \emph{not} a valid command.
    \begin{verbatim}
    ./myworm 192.168.10.0 192.168.0.0
    \end{verbatim}.

  \item The worm process runs in the background. Specifically, it
    calls a Linux API, called \texttt{daemon}. See the manual
    (\texttt{man daemon}) for more information. You do not need to use
    the \texttt{daemon} API if you prefer to use \texttt{fork} and
    \texttt{setsid}.

\end{enumerate}

\section{Competition (Extra Points)}

We will pick the most well-designed worm, and use it in the next
activity. Furthermore, we will give +10 extra points to the student!

\section{Final Deliverables}

We expect you to push your source code to your own private repository
(worm-your\_id). Do not push your binaries to the repository. You
\emph{must} follow the file naming convention described below.
\textbf{If otherwise, course staffs will immediately deduct half of
the available score}.

\begin{enumerate}

  \item \textbf{./Makefile} file is to build your worm. When we type
    ``make'', it should automatically build your program and place the
    \texttt{myworm} binary to \texttt{./build/myworm}. Again, you can
    use any language to implement your worm, but the final binary
    executable should be placed in \texttt{./build/myworm} after
    typing ``make''. N.B. Never commit the binary to the repository.
    We will immediately deduct 3 points if you commit a binary itself
    to the build directory.

  \item \textbf{./src/} directory contains the source of your worm.
    You can use any language you want. N.B. Points will be deducted
    for the lack of necessary comments. Please make your code
    readable.

  \item \textbf{./README.md} file contains a write-up. This write-up
    simply lists what you did for each directory and file, and what
    you learned from this activity.

\end{enumerate}

\bibliography{references}
\bibliographystyle{plainnat}

\end{document}
